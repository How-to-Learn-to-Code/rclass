%%%%%%%%%%%%%%%%%%%%%%%%%%%%%%%%%%%%%%%%%
% Short Sectioned Assignment
% LaTeX Template
% Version 1.0 (5/5/12)
%
% This template has been downloaded from:
% http://www.LaTeXTemplates.com
%
% Original author:
% Frits Wenneker (http://www.howtotex.com)
%
% License:
% CC BY-NC-SA 3.0 (http://creativecommons.org/licenses/by-nc-sa/3.0/)
%
%%%%%%%%%%%%%%%%%%%%%%%%%%%%%%%%%%%%%%%%%

%----------------------------------------------------------------------------------------
%	PACKAGES AND OTHER DOCUMENT CONFIGURATIONS
%----------------------------------------------------------------------------------------

\documentclass[paper=a4, fontsize=11pt]{scrartcl} % A4 paper and 11pt font size

\usepackage[T1]{fontenc} % Use 8-bit encoding that has 256 glyphs
\usepackage{fourier} % Use the Adobe Utopia font for the document - comment this line to return to the LaTeX default
\usepackage[english]{babel} % English language/hyphenation
\usepackage{amsmath,amsfonts,amsthm} % Math packages
\usepackage{multirow}
\usepackage[normalem]{ulem}
\useunder{\uline}{\ul}{}

\usepackage{lipsum} % Used for inserting dummy 'Lorem ipsum' text into the template

\usepackage{sectsty} % Allows customizing section commands
\allsectionsfont{\centering \normalfont\scshape} % Make all sections centered, the default font and small caps

\usepackage{fancyhdr} % Custom headers and footers
\pagestyle{fancyplain} % Makes all pages in the document conform to the custom headers and footers
\fancyhead{} % No page header - if you want one, create it in the same way as the footers below
\fancyfoot[L]{} % Empty left footer
\fancyfoot[C]{} % Empty center footer
\fancyfoot[R]{\thepage} % Page numbering for right footer
\renewcommand{\headrulewidth}{0pt} % Remove header underlines
\renewcommand{\footrulewidth}{0pt} % Remove footer underlines
\setlength{\headheight}{13.6pt} % Customize the height of the header

\numberwithin{equation}{section} % Number equations within sections (i.e. 1.1, 1.2, 2.1, 2.2 instead of 1, 2, 3, 4)
\numberwithin{figure}{section} % Number figures within sections (i.e. 1.1, 1.2, 2.1, 2.2 instead of 1, 2, 3, 4)
\numberwithin{table}{section} % Number tables within sections (i.e. 1.1, 1.2, 2.1, 2.2 instead of 1, 2, 3, 4)

\setlength\parindent{0pt} % Removes all indentation from paragraphs - comment this line for an assignment with lots of text

%----------------------------------------------------------------------------------------
%	TITLE SECTION
%----------------------------------------------------------------------------------------

\newcommand{\horrule}[1]{\rule{\linewidth}{#1}} % Create horizontal rule command with 1 argument of height

\title{	
\normalfont \normalsize 
\textsc{How to Learn to Code} \\ [25pt] % Your university, school and/or department name(s)
\horrule{0.5pt} \\[0.4cm] % Thin top horizontal rule
\huge R Syllabus \\ % The assignment title
\horrule{2pt} \\[0.5cm] % Thick bottom horizontal rule
}

\author{Amy Pomeroy} % ADD YOUR NAME HERE IF YOU CONTRIBUTE!! 

\date{\normalsize\today} % Today's date or a custom date

\begin{document}

\maketitle % Print the title

%----------------------------------------------------------------------------------------
%	COURSE LEARNING OBJECTIVES
%----------------------------------------------------------------------------------------

\section{Course Leanring Objectives}

At the end of this course students will be able to:

\begin{enumerate}
\item import and export data
\item write code to solve appropriately difficult problems 
\item generate plots of experimental data
\item independently use resources to solve bugs
\item explain how coding can be used to help process as manage biologically/medically relevant data
\end{enumerate}

%----------------------------------------------------------------------------------------
%	CLASS SCHEDULE
%----------------------------------------------------------------------------------------

\section{Class Schedule}

Below is a list the classes and their learning objectives \\ \\

\textbf{Set-up and the basics} \\
Learning objective: Students will be comfortable doing basic commands in the console. \\
Specific coding skills:
\begin{itemize}
	\item{basic math operators (+, -, *, /)}
	\item{assignment operator (<-)}
	\item{basic functions (ls(), class(), is(), as(), c())}
	\item{common data classes (character, numeric, logical)}
	\item{comparison operators (`>`, `<`, `==`, `>=`, `<=`)}
\end{itemize}  

\textbf{Data structures and subsetting} \\
Learning objectives:  Students will be able to compare and contrast basic data structures. \\
Specific coding skills:
\begin{itemize}
	\item{basic data structures (vectors, matrices, lists, data frames)}
	\item{subsetting for each data structure (vectors, matrices, lists, data frames)}
\end{itemize}  

\textbf{Subsetting and introduction to plotting} \\
Learning objectives: Students will be able to import and export data from and to csv files \\
Specific coding skills:
\begin{itemize}
	\item{file paths and making new files and directories (file.path(), dir.create())}
	\item{import data from csv file (read.csv())}
	\item{save data (save())}
	\item{export data to csv file (write.csv())}
\end{itemize}  

\textbf{Plotting} \\
Learning objectives: Students will be able to generate different types of plots to represent experimental data. \\
Specific coding skills:
\begin{itemize}
	\item{basic scatter plot (plot())}
	\item{additional plotting functions (lines(), points(), hist(), barplot(), boxplot(), pairs())}
	\item{making legend (legend())}
\end{itemize}  

\textbf{Control} \\
Learning objectives: Students will be able to apply control statements. \\
Specific coding skills:
\begin{itemize}
	\item{conditionals (if(), else()}
	\item{loops (for(), while())}
\end{itemize}  

\textbf{Functions} \\
Learning objectives: Students will be able to write and run a function. \\
Specific coding skills:
\begin{itemize}
	\item{function notation}
	\item{function environment}
	\item{don't repeat yourself (DRY) principle}
	\item{functional control statements (return(), warning(), stop())}
\end{itemize}  

\textbf{Packages} \\
Learning objectives: Students will be able to create a simple R package using RStudio and roxygen2. \\
Specific coding skills:
\begin{itemize}
	\item{installing and loading R packages (install.packages(), library())}
	\item{basic components of a package}
	\item{creating an R package}
\end{itemize}  


 %----------------------------------------------------------------------------------------
%	ASSESSMENT
%----------------------------------------------------------------------------------------

\section{Assessment}

This class is not for a grade, but you will get out of it what you put into it. Opportunities to assess your understanding of the material will be provided throughout the course. 

%------------------------------------------------

\subsection{In class assignments and homework}

You will have assignments during each class period. If you do not finish them in class time then you can finish them as homework. These will not be graded, but there will be an opportunity to discuss any questions at the beginning of the next class period. Answer keys will be provided before the next class period. 

\subsection{Final assessment}

In order to get the certificate there will be a final assessment. This assessment will be sent to you the last week of class along with a feedback survey. It will take less than an hour. As long as you make your best effort on the assessment, you will get credit for the course. The assessment's purpose is to help the instructors improve the curriculum for future years. 

%----------------------------------------------------------------------------------------
%	EXPECTATIONS
%----------------------------------------------------------------------------------------

\section{Expectations}

\textbf{What teachers expect of students}\\
Teachers expect students to...
\begin{itemize}
\item be on time for class, and email them if they are going to miss a class
\item complete the in class assignments
\item ask questions when they don't understand
\item be respectful of people of all different backgrounds and abilities 
\end{itemize}

\textbf{What students can expect of teachers}\\
Students can expect teachers to...
\begin{itemize}
\item be on time and prepared for class
\item answer questions in class and via email (within one business day)
\item be respectful of people of all different backgrounds and abilities 
\end{itemize}

\end{document}